% !TEX program = xelatex
\documentclass[11pt,a4paper]{article}

\usepackage{fontspec}
\usepackage{xunicode,xltxtra}
\XeTeXlinebreaklocale "th"
\XeTeXlinebreakskip = 0pt plus 2pt minus 1pt
\setmainfont[
  UprightFont={Laksaman.otf},
  ItalicFont={Laksaman-Italic.otf},
  BoldFont={Laksaman-Bold.otf},
  BoldItalicFont={Laksaman-BoldItalic.otf},
  Script=Thai,
  Scale=MatchLowercase,
  WordSpace=1.25
]{Laksaman.otf}

\input{header}
\usepackage[unicode=true]{hyperref}
\hypersetup{
  colorlinks=true,
  linkcolor=black,
  citecolor=blue!50!black,
  urlcolor=blue!80!black
}

% --- Thai caption labels ---
\usepackage{caption}
\captionsetup{font=small,labelfont=bf}
\renewcommand{\figurename}{รูปที่}
\renewcommand{\tablename}{ตารางที่}

% --- Thai autoref labels ---
\renewcommand*{\figureautorefname}{รูปที่}
\renewcommand*{\tableautorefname}{ตารางที่}
\renewcommand*{\sectionautorefname}{หัวข้อ}
\renewcommand*{\subsectionautorefname}{หัวข้อย่อย}

\usepackage[a4paper,top=1in,bottom=1in,left=1in,right=1in]{geometry}
\usepackage{graphicx}
\usepackage{fancyhdr}
\usepackage{longtable,booktabs,array,listings,xcolor,caption,float,enumitem} \usepackage{setspace} 
\usepackage[unicode=true]{hyperref} 
\hypersetup{colorlinks=true,linkcolor=black,citecolor=blue!50!black,urlcolor=blue!80!black} \lstdefinestyle{code}{basicstyle=\ttfamily\small,numbers=left,numberstyle=\tiny,stepnumber=1,numbersep=5pt, showstringspaces=false,breaklines=true,frame=single,tabsize=2} 

\newcommand{\hrsp}{\kern0.0625em} 
\setstretch{1.35}
% --- Listings Languages Setup ---
\usepackage{listings}
\usepackage{xcolor}

% โหลดภาษา
% --- JavaScript / TypeScript ---
\definecolor{jskeyword}{RGB}{0,102,204}
\definecolor{jsstring}{RGB}{163,21,21}
\definecolor{jscomment}{RGB}{0,128,0}
\definecolor{jsnumber}{RGB}{128,0,128}
\definecolor{jsbg}{RGB}{248,248,248}

\lstdefinelanguage{JavaScript}{
  keywords={break,case,catch,class,const,continue,debugger,default,delete,do,else,export,extends,
  finally,for,function,if,import,in,instanceof,new,return,super,switch,this,throw,try,
  typeof,var,void,while,with,yield,let,static,async,await},
  keywordstyle=\color{jskeyword}\bfseries,
  comment=[l]{//},
  morecomment=[s]{/*}{*/},
  commentstyle=\color{jscomment}\itshape,
  stringstyle=\color{jsstring},
  morestring=[b]', morestring=[b]",
  numberstyle=\tiny\color{gray},
  backgroundcolor=\color{jsbg}
}

\lstdefinelanguage{TypeScript}[]{JavaScript}{
  morekeywords={type,interface,readonly,enum,implements,namespace,abstract,declare}
}

% --- SQL ---
\definecolor{sqlkeyword}{RGB}{0,0,150}
\definecolor{sqlstring}{RGB}{150,0,0}
\definecolor{sqlcomment}{RGB}{0,128,0}
\definecolor{sqlbg}{RGB}{248,248,248}

\lstdefinelanguage{SQL}{
  morekeywords={
    SELECT,FROM,WHERE,AND,OR,NOT,INSERT,INTO,VALUES,UPDATE,SET,DELETE,JOIN,LEFT,RIGHT,FULL,
    OUTER,INNER,ON,AS,CREATE,DROP,ALTER,TABLE,VIEW,INDEX,PRIMARY,KEY,FOREIGN,REFERENCES,
    GROUP,BY,ORDER,HAVING,DISTINCT,TOP,COUNT,SUM,AVG,MIN,MAX,NULL,IS,IN,EXISTS,BETWEEN,CASE,WHEN,THEN,END
  },
  keywordstyle=\color{sqlkeyword}\bfseries,
  comment=[l]{--},
  morecomment=[s]{/*}{*/},
  commentstyle=\color{sqlcomment}\itshape,
  stringstyle=\color{sqlstring},
  morestring=[b]', morestring=[b]",
  backgroundcolor=\color{sqlbg}
}

% --- C# ---
\definecolor{cskeyword}{RGB}{0,0,180}
\definecolor{csstring}{RGB}{163,21,21}
\definecolor{cscomment}{RGB}{0,128,0}
\definecolor{csbg}{RGB}{248,248,248}

\lstdefinelanguage{CSharp}{
  morekeywords={
    abstract,as,base,bool,break,byte,case,catch,char,checked,class,const,continue,decimal,
    default,delegate,do,double,else,enum,event,explicit,extern,false,finally,fixed,float,
    for,foreach,goto,if,implicit,in,int,interface,internal,is,lock,long,namespace,new,null,
    object,operator,out,override,params,private,protected,public,readonly,ref,return,sbyte,
    sealed,short,sizeof,stackalloc,static,string,struct,switch,this,throw,true,try,typeof,
    uint,ulong,unchecked,unsafe,ushort,using,virtual,void,volatile,while,var,dynamic,async,await
  },
  keywordstyle=\color{cskeyword}\bfseries,
  comment=[l]{//},
  morecomment=[s]{/*}{*/},
  commentstyle=\color{cscomment}\itshape,
  stringstyle=\color{csstring},
  morestring=[b]', morestring=[b]",
  backgroundcolor=\color{csbg}
}


% --- Default Code Style ---
\lstdefinestyle{code}{
  basicstyle=\ttfamily\small,
  numbers=left, numberstyle=\tiny, stepnumber=1, numbersep=5pt,
  showstringspaces=false, breaklines=true, frame=single,
  tabsize=2, backgroundcolor=\color{gray!3}
}

% --- Listings Languages Setup ---
\usepackage{listings}
\usepackage{xcolor}

% โหลดภาษา
% --- JavaScript / TypeScript ---
\definecolor{jskeyword}{RGB}{0,102,204}
\definecolor{jsstring}{RGB}{163,21,21}
\definecolor{jscomment}{RGB}{0,128,0}
\definecolor{jsnumber}{RGB}{128,0,128}
\definecolor{jsbg}{RGB}{248,248,248}

\lstdefinelanguage{JavaScript}{
  keywords={break,case,catch,class,const,continue,debugger,default,delete,do,else,export,extends,
  finally,for,function,if,import,in,instanceof,new,return,super,switch,this,throw,try,
  typeof,var,void,while,with,yield,let,static,async,await},
  keywordstyle=\color{jskeyword}\bfseries,
  comment=[l]{//},
  morecomment=[s]{/*}{*/},
  commentstyle=\color{jscomment}\itshape,
  stringstyle=\color{jsstring},
  morestring=[b]', morestring=[b]",
  numberstyle=\tiny\color{gray},
  backgroundcolor=\color{jsbg}
}

\lstdefinelanguage{TypeScript}[]{JavaScript}{
  morekeywords={type,interface,readonly,enum,implements,namespace,abstract,declare}
}

% --- SQL ---
\definecolor{sqlkeyword}{RGB}{0,0,150}
\definecolor{sqlstring}{RGB}{150,0,0}
\definecolor{sqlcomment}{RGB}{0,128,0}
\definecolor{sqlbg}{RGB}{248,248,248}

\lstdefinelanguage{SQL}{
  morekeywords={
    SELECT,FROM,WHERE,AND,OR,NOT,INSERT,INTO,VALUES,UPDATE,SET,DELETE,JOIN,LEFT,RIGHT,FULL,
    OUTER,INNER,ON,AS,CREATE,DROP,ALTER,TABLE,VIEW,INDEX,PRIMARY,KEY,FOREIGN,REFERENCES,
    GROUP,BY,ORDER,HAVING,DISTINCT,TOP,COUNT,SUM,AVG,MIN,MAX,NULL,IS,IN,EXISTS,BETWEEN,CASE,WHEN,THEN,END
  },
  keywordstyle=\color{sqlkeyword}\bfseries,
  comment=[l]{--},
  morecomment=[s]{/*}{*/},
  commentstyle=\color{sqlcomment}\itshape,
  stringstyle=\color{sqlstring},
  morestring=[b]', morestring=[b]",
  backgroundcolor=\color{sqlbg}
}

% --- C# ---
\definecolor{cskeyword}{RGB}{0,0,180}
\definecolor{csstring}{RGB}{163,21,21}
\definecolor{cscomment}{RGB}{0,128,0}
\definecolor{csbg}{RGB}{248,248,248}

\lstdefinelanguage{CSharp}{
  morekeywords={
    abstract,as,base,bool,break,byte,case,catch,char,checked,class,const,continue,decimal,
    default,delegate,do,double,else,enum,event,explicit,extern,false,finally,fixed,float,
    for,foreach,goto,if,implicit,in,int,interface,internal,is,lock,long,namespace,new,null,
    object,operator,out,override,params,private,protected,public,readonly,ref,return,sbyte,
    sealed,short,sizeof,stackalloc,static,string,struct,switch,this,throw,true,try,typeof,
    uint,ulong,unchecked,unsafe,ushort,using,virtual,void,volatile,while,var,dynamic,async,await
  },
  keywordstyle=\color{cskeyword}\bfseries,
  comment=[l]{//},
  morecomment=[s]{/*}{*/},
  commentstyle=\color{cscomment}\itshape,
  stringstyle=\color{csstring},
  morestring=[b]', morestring=[b]",
  backgroundcolor=\color{csbg}
}


% --- Default Code Style ---
\lstdefinestyle{code}{
  basicstyle=\ttfamily\small,
  numbers=left, numberstyle=\tiny, stepnumber=1, numbersep=5pt,
  showstringspaces=false, breaklines=true, frame=single,
  tabsize=2, backgroundcolor=\color{gray!3}
}


% Define degree and micro symbols if not available
\providecommand{\textdegree}{\ensuremath{^\circ}}
\providecommand{\textmu}{\ensuremath{\mu}}

% Prolog syntax highlighting
\lstdefinestyle{prologstyle}{
    language=Prolog,
    basicstyle=\ttfamily\small,
    keywordstyle=\color{blue}\bfseries,
    commentstyle=\color{green!50!black},
    stringstyle=\color{red},
    showstringspaces=false,
    breaklines=true,
    frame=single,
    numbers=left,
    numberstyle=\tiny\color{gray}
}

\lstdefinestyle{pythonstyle}{
    language=Python,
    basicstyle=\ttfamily\small,
    keywordstyle=\color{blue}\bfseries,
    commentstyle=\color{green!50!black},
    stringstyle=\color{red},
    showstringspaces=false,
    breaklines=true,
    frame=single,
    numbers=left,
    numberstyle=\tiny\color{gray}
}

\title{ระบบติดป้ายกำกับข้อมูลอัตโนมัติด้วย AI\\โดยใช้ Prolog และ Large Language Models}
\author{ระบบ Auto-Labeling Engine สำหรับข้อมูลตารางทั่วไป}
\date{\today}

\begin{document}

\maketitle

\begin{abstract}
บทความนี้นำเสนอระบบติดป้ายกำกับข้อมูลอัตโนมัติที่ผลรวมความสามารถในการให้เหตุผลของ Prolog เข้ากับความเข้าใจภาษาธรรมชาติของ Large Language Models (LLMs) ระบบนี้ช่วยให้ผู้เชี่ยวชาญในสาขาต่างๆ สามารถแสดงกฎการติดป้ายกำกับด้วยภาษาธรรมชาติ ซึ่งจะถูกแปลงเป็นประพจน์ First-Order Logic Prolog โดยอัตโนมัติผ่าน Gemini API จากนั้นกฎเหล่านี้จะถูกนำไปใช้กับข้อมูลในรูปแบบ CSV โดยใช้ PySwip ซึ่งเป็นตัวเชื่อมระหว่าง Python และ Prolog ระบบสามารถรองรับกฎแบบลูกโซ่ (Chain Rules) สำหรับตัวตรรกะที่ซับซ้อน การจำแนกหลายป้ายกำกับพร้อมกัน (Multi-label) และมีระบบแสดงผลกราฟอัตโนมัติ เรานำเสนอ Auto-Labeling Engine ใหม่ที่ออกแบบมาเพื่อรองรับข้อมูลตารางทั่วไป (Tabular Data) และสามารถปรับใช้กับโดเมนต่างๆ ได้อย่างง่ายดายผ่านการตั้งค่า JSON โดยไม่ต้องเปลี่ยนโค้ด
\end{abstract}

\section{บทนำ}

การติดป้ายกำกับข้อมูล (Data Labeling) เป็นงานที่สำคัญแต่ใช้แรงงานมากในกระบวนการเตรียมข้อมูลสำหรับการเรียนรู้ของเครื่อง (Machine Learning) และการวิเคราะห์ข้อมูล วิธีการแบบดั้งเดิมต้องอาศัยการอธิบายด้วยตนเองจากผู้เชี่ยวชาญหรือใช้ระบบกฎที่ต้องเขียนโปรแกรมที่ซับซ้อน งานวิจัยนี้นำเสนอแนวทางแบบผสมผสาน (Hybrid Approach) ที่:

\begin{enumerate}
    \item รับคำอธิบายกฎในรูปแบบภาษาธรรมชาติจากผู้เชี่ยวชาญ
    \item แปลงเป็นตรรกะเชิงรูปแบบ (Formal Logic) โดยอัตโนมัติด้วย Prolog
    \item ประยุกต์ใช้กฎเหล่านี้กับข้อมูลขนาดใหญ่อย่างสม่ำเสมอ
    \item รองรับการให้เหตุผลที่ซับซ้อนผ่าน Chain Rules
    \item ให้ข้อมูลป้อนกลับและการตรวจสอบผ่านภาพ (Visual Feedback)
\end{enumerate}

\subsection{Auto-Labeling Engine ที่พัฒนาขึ้นใหม่}

เราได้พัฒนา Auto-Labeling Engine ที่ออกแบบมาโดยเฉพาะสำหรับข้อมูลตารางทั่วไป (Generic Tabular Data) โดย Engine นี้มีคุณสมบัติหลักดังนี้:

\begin{itemize}
    \item \textbf{Configuration-Driven Architecture}: สามารถปรับใช้กับข้อมูลชุดใหม่ได้โดยแก้ไขแค่ไฟล์ JSON โดยไม่ต้องเปลี่ยนโค้ด
    \item \textbf{Point-based Inference}: ประมวลผลการให้เหตุผล (Inference) แบบ Row-by-Row โดยพิจารณาค่าของแต่ละจุดข้อมูลในขณะนั้น
    \item \textbf{Dynamic Prolog Integration}: โหลดและประมวลผลกฎ Prolog แบบไดนามิก รองรับ Chain Rules และ Helper Predicates
    \item \textbf{Multi-domain Support}: รองรับหลากหลายสาขา เช่น การตรวจสอบคุณภาพอากาศ, การติดตามสุขภาพ, การควบคุมคุณภาพในอุตสาหกรรม
    \item \textbf{Automatic Visualization}: สร้างกราฟแสดงผลอัตโนมัติสำหรับการวิเคราะห์ข้อมูลเชิงเวลา (Time-series Analysis)
\end{itemize}

ความแตกต่างจากระบบอื่นๆ คือ Engine ของเราเน้นความง่ายในการใช้งานสำหรับผู้ที่ไม่มีความรู้ด้านการเขียนโปรแกรม โดยใช้ภาษาธรรมชาติเป็นอินเทอร์เฟซหลัก และรักษาความสามารถในการอธิบายได้ (Explainability) ผ่าน Prolog Rules ที่มนุษย์อ่านเข้าใจได้

\section{สถาปัตยกรรมระบบ}

\subsection{องค์ประกอบหลัก}

ระบบประกอบด้วย 4 ส่วนหลัก:

\begin{enumerate}
    \item \textbf{ส่วนติดต่อผู้ใช้แบบภาษาธรรมชาติ (Natural Language Interface)}: ใช้ Tkinter GUI รับข้อมูลจากผู้ใช้
    \item \textbf{ตัวแปลภาษาด้วย LLM (LLM Translation Engine)}: ใช้ Gemini API แปลงภาษาธรรมชาติเป็นกฎ Prolog
    \item \textbf{ตัวประมวลผลตรรกะ Prolog (Prolog Inference Engine)}: ใช้ PySwip นำกฎไปใช้กับข้อมูล
    \item \textbf{ส่วนแสดงผลกราฟ (Visualization Module)}: ใช้ Matplotlib แสดงผลการวิเคราะห์ข้อมูลเชิงเวลา
\end{enumerate}

\subsection{ขั้นตอนการทำงาน}

\begin{enumerate}
    \item ผู้ใช้ป้อนกฎเป็นภาษาธรรมชาติ (ไทยหรืออังกฤษ)
    \item ระบบโหลดการตั้งค่าสำหรับ Use Case ที่เลือก
    \item Gemini API แปลงเป็นประพจน์ Prolog
    \item ตรวจสอบและบันทึกกฎพร้อม Timestamp
    \item PySwip โหลดกฎและ Query แต่ละแถวข้อมูล
    \item ติดป้ายกำกับและบันทึกลง CSV พร้อม Metadata
    \item แสดงผลรูปแบบกราฟและการกระจายตัวของป้ายกำกับ
\end{enumerate}

\subsection{Auto-Labeling Engine Architecture}

Engine ที่เราพัฒนาขึ้นมีสถาปัตยกรรมแบบ Modular ที่แยกส่วนการทำงานออกเป็นโมดูลต่างๆ:

\begin{itemize}
    \item \textbf{Configuration Manager} (\texttt{query\_engine\_config.py}): จัดการการตั้งค่าทั้งหมด รวมทั้ง path, ชื่อไฟล์, และการแมปตัวแปร
    \item \textbf{Rule Parser} (\texttt{extract\_predicates\_from\_rules}): แยกวิเคราะห์กฎ Prolog เพื่อหาชื่อ Predicate, จำนวนและชื่อของ Arguments
    \item \textbf{Data Mapper}: แปลงข้อมูลจาก CSV เป็นรูปแบบที่ Prolog เข้าใจได้ (เช่น Time ``HH:MM'' $\rightarrow$ Integer Hour)
    \item \textbf{Inference Engine} (\texttt{apply\_rule\_to\_csv}): ประมวลผลแต่ละแถวด้วย Point-based Inference
    \item \textbf{Visualization Generator} (\texttt{plot\_labeled\_results}): สร้างกราฟแสดงผลอัตโนมัติ
\end{itemize}

\section{วิธีการ (Methodology)}

\subsection{การแปลงภาษาธรรมชาติเป็น Prolog}

ระบบใช้ Prompt Template ที่ออกแบบมาอย่างละเอียดซึ่งระบุ:

\begin{itemize}
    \item ไวยากรณ์ Prolog และตัวดำเนินการที่ถูกต้อง ($\leq$, $\geq$, $>$, $<$)
    \item รูปแบบ Predicate แบบหลายอาร์กิวเมนต์: \texttt{predicate(Var1, Var2, ..., 'label') :- conditions}
    \item กฎการตั้งชื่อตัวแปร (ตัวแปรขึ้นต้นด้วยตัวพิมพ์ใหญ่, predicates เป็นตัวพิมพ์เล็ก)
    \item รูปแบบ Chain Rules สำหรับตรรกะที่ซับซ้อน
\end{itemize}

\textbf{ตัวอย่างการแปลง:}

\textit{Input (ภาษาธรรมชาติ):}
\begin{quote}
``จำแนกคุณภาพอากาศเป็น 3 ระดับ: ดี (PM2.5 < 25), ปานกลาง (PM2.5 25-50), แย่ (PM2.5 > 50)''
\end{quote}

\textit{Output (Prolog):}
\begin{lstlisting}[style=prologstyle]
good_pm(PM2_5) :- PM2_5 < 25.
moderate_pm(PM2_5) :- PM2_5 >= 25, PM2_5 =< 50.
poor_pm(PM2_5) :- PM2_5 > 50.

label_air_quality(PM2_5, 'ดี') :- good_pm(PM2_5).
label_air_quality(PM2_5, 'ปานกลาง') :- moderate_pm(PM2_5).
label_air_quality(PM2_5, 'แย่') :- poor_pm(PM2_5).
\end{lstlisting}

\subsection{การรองรับ Chain Rules}

ระบบรองรับการให้เหตุผลแบบลำดับชั้นผ่าน Helper Predicates:

\begin{lstlisting}[style=prologstyle]
% Helper predicates (ไม่มี label argument)
high_pm(PM2_5) :- PM2_5 > 50.
high_temp(Temperature) :- Temperature > 25.

% กฎการติดป้ายกำกับหลัก (ใช้ helpers)
label_condition(PM2_5, Temperature, 'ต้องเฝ้าระวัง') :- 
    high_pm(PM2_5), high_temp(Temperature).
\end{lstlisting}

\subsection{กระบวนการใช้กฎ (Point-based Inference)}

สำหรับแต่ละแถวในชุดข้อมูล:

\begin{enumerate}
    \item แยกค่าและแมปกับตัวแปร Prolog
    \item แปลงชนิดข้อมูล (เช่น ``HH:MM'' $\rightarrow$ ชั่วโมงเป็นจำนวนเต็ม)
    \item Assert facts เข้าไปใน Prolog Knowledge Base
    \item Query ทุก Label Predicates ด้วยค่าของแถวปัจจุบัน
    \item รวบรวมป้ายกำกับที่ตรงเงื่อนไข (single-label หรือ multi-label)
    \item Retract facts เพื่อเตรียมสำหรับแถวถัดไป
\end{enumerate}

\textbf{หมายเหตุสำคัญ}: กระบวนการนี้เป็นแบบ \textit{stateless} คือแต่ละแถวถูกประมวลผลอิสระจากกัน ไม่มีการเก็บข้อมูลจากแถวก่อนหน้า ดังนั้น \textbf{ไม่สามารถใช้กฎที่ต้องการเปรียบเทียบกับข้อมูลย้อนหลังได้} เช่น:
\begin{itemize}
    \item ``ถ้าค่าลดลง 5 หน่วยจากแถวก่อนหน้า''
    \item ``ถ้าค่าเฉลี่ยของ 3 แถวล่าสุดมากกว่า 50''
\end{itemize}

\section{Implementation Details}

\subsection{Configuration Management}

The system uses JSON configuration files for each use case:

\begin{lstlisting}[style=pythonstyle]
{
  "use_case": "PM_Temperature",
  "prolog_variables": [
    {"csv_column": "Temp", "prolog_name": "Temperature"},
    {"csv_column": "PM2.5", "prolog_name": "PM2_5"},
    {"csv_column": "Time", "prolog_name": "TimeHour"}
  ],
  "labeling": {
    "label_column": "auto_label",
    "multi_label": true
  }
}
\end{lstlisting}

\subsection{Argument Parsing}

Critical feature: The system parses Prolog rules to extract argument names and order:

\begin{lstlisting}[style=pythonstyle]
def extract_predicates_from_rules(rule_file):
    # Parse: predicate(Arg1, Arg2, ..., 'Label') :- condition
    match = re.search(r"(\w+)\((.+),\s*'([^']+)'\)\s*:-", line)
    arg_list = [a.strip() for a in args_str.split(',')]
    # Handles underscore variables: _, _Var
\end{lstlisting}

\subsection{Visualization}

The system automatically generates two-panel visualizations:

\begin{enumerate}
    \item \textbf{Time Series Plot}: Shows PM2.5 and Temperature trends with labeled regions highlighted as colored spans
    \item \textbf{Distribution Chart}: Bar chart showing frequency of each label
\end{enumerate}

Consecutive data points with the same label are grouped into shaded regions, reducing visual clutter.

\section{กรณีศึกษา: ระบบตรวจสอบคุณภาพอากาศ PM2.5}

\subsection{ชุดข้อมูล}

เราใช้ข้อมูลคุณภาพอากาศเป็นตัวอย่างการใช้งานระบบ (ไม่ใช่จุดเน้นหลักของ Engine):

\begin{itemize}
    \item \textbf{แหล่งข้อมูล}: การวัด 72 ชั่วโมงต่อเนื่อง (3 วัน)
    \item \textbf{Features}: อุณหภูมิ (\textdegree C), PM2.5 (\textmu g/m\textsuperscript{3}), เวลา (HH:MM)
    \item \textbf{วัตถุประสงค์}: จำแนกสภาพคุณภาพอากาศ
\end{itemize}

\textbf{หมายเหตุ}: กรณีศึกษานี้เป็นเพียงตัวอย่างหนึ่งของการประยุกต์ใช้ Engine ที่เราพัฒนาขึ้น ระบบสามารถใช้กับข้อมูลตารางทั่วไปในสาขาอื่นๆ ได้เช่นกัน เช่น:
\begin{itemize}
    \item การติดตามสัญญาณชีพของผู้ป่วย (อุณหภูมิร่างกาย, ชีพจร, ความดันโลหิต)
    \item การตรวจสอบคุณภาพผลิตภัณฑ์ในโรงงาน (ขนาด, น้ำหนัก, อุณหภูมิ)
    \item การวิเคราะห์ข้อมูลการเงิน (ราคา, ปริมาณการซื้อขาย, ตัวชี้วัดทางเทคนิค)
\end{itemize}

\subsection{ตัวอย่างกฎ}

\textbf{กฎที่ 1 - การจำแนกแบบง่าย:}
\begin{lstlisting}[style=prologstyle]
air_quality(PM2_5, 'ดี') :- PM2_5 < 25.
air_quality(PM2_5, 'ปานกลาง') :- PM2_5 >= 25, PM2_5 =< 50.
air_quality(PM2_5, 'แย่') :- PM2_5 > 50.
\end{lstlisting}

\textbf{กฎที่ 2 - Multi-Variable พร้อม Chain Rules:}
\begin{lstlisting}[style=prologstyle]
poor_pm(PM2_5) :- PM2_5 > 50.
high_temp_for_monitoring(Temperature) :- Temperature > 25.

label_overall_condition(PM2_5, Temperature, 'ต้องเฝ้าระวัง') :- 
    poor_pm(PM2_5), 
    high_temp_for_monitoring(Temperature).
\end{lstlisting}

\subsection{ผลลัพธ์}

\begin{table}[h]
\centering
\begin{tabular}{@{}lcc@{}}
\toprule
\textbf{ป้ายกำกับ} & \textbf{จำนวน} & \textbf{เปอร์เซ็นต์} \\ \midrule
ปานกลาง & 37 & 51.4\% \\
แย่ & 35 & 48.6\% \\
ดี & 0 & 0\% \\ \bottomrule
\end{tabular}
\caption{การกระจายตัวของป้ายกำกับคุณภาพอากาศ PM2.5}
\end{table}

ระบบระบุช่วงเวลาที่มีคุณภาพอากาศแย่ (PM2.5 > 50) และปานกลางได้สำเร็จ โดยไม่มีการอ่านค่า ``ดี'' ในช่วงมลพิษนี้

\subsection{ข้อสังเกตสำคัญ}

กฎเหล่านี้เป็น \textbf{Point-based Rules} คือพิจารณาเฉพาะค่า ณ เวลานั้นๆ หากต้องการกฎที่ซับซ้อนขึ้น เช่น:
\begin{itemize}
    \item ``ถ้า PM2.5 เพิ่มขึ้น 20 หน่วยภายใน 2 ชั่วโมง ให้เตือนว่ามลพิษเพิ่มขึ้นอย่างรวดเร็ว''
\end{itemize}

Engine ปัจจุบัน\textbf{ยังไม่รองรับ}เพราะต้องการข้อมูลย้อนหลัง ซึ่งเป็นส่วนหนึ่งของงานพัฒนาในอนาคต

\section{คุณสมบัติหลัก}

\subsection{การจัดเก็บกฎแบบ Timestamped}

กฎที่สร้างขึ้นทั้งหมดจะถูกบันทึกพร้อม Timestamp:
\begin{verbatim}
generated_rules_20251205_011531.pl
\end{verbatim}

วิธีนี้ช่วยเก็บประวัติและเปรียบเทียบชุดกฎต่างๆ ได้

\subsection{การรองรับ Multi-Label}

ระบบสามารถติดหลายป้ายกำกับพร้อมกันได้:
\begin{verbatim}
แถวที่ 34: "แย่; ต้องเฝ้าระวัง"
\end{verbatim}

\subsection{การติดตาม Metadata}

CSV ที่ผลลัพธ์จะมี:
\begin{itemize}
    \item คอลัมน์ข้อมูลเดิม
    \item \texttt{auto\_label}: ป้ายกำกับที่ถูกติด
    \item \texttt{rules\_file}: ชื่อไฟล์กฎที่ถูกใช้
\end{itemize}

\subsection{Configuration-Driven Design}

ผู้ใช้สามารถปรับใช้กับข้อมูลชุดใหม่ได้โดยแก้ไขเฉพาะไฟล์ JSON:
\begin{itemize}
    \item กำหนด CSV columns และ Prolog variable mapping
    \item ระบุ path สำหรับไฟล์ input/output
    \item เลือก single-label หรือ multi-label mode
    \item กำหนด prompt template เป็นภาษาไทยหรืออังกฤษ
\end{itemize}

ไม่ต้องเปลี่ยนโค้ด Python เลย - เพียงแค่สร้าง use case ใหม่ในโฟลเดอร์ \texttt{KB/}

\section{ข้อดี}

\begin{enumerate}
    \item \textbf{ใช้งานง่าย}: ผู้เชี่ยวชาญในสาขาต่างๆ สามารถเขียนกฎด้วยภาษาธรรมชาติได้
    \item \textbf{อธิบายได้ (Explainable)}: กฎ Prolog อ่านและตรวจสอบได้โดยมนุษย์
    \item \textbf{สม่ำเสมอ (Consistent)}: กฎเดียวกันถูกใช้กับข้อมูลทั้งหมดอย่างเท่าเทียม
    \item \textbf{ยืดหยุ่น (Flexible)}: รองรับตรรกะที่ซับซ้อนผ่าน Chain Rules
    \item \textbf{ตรวจสอบย้อนกลับได้ (Traceable)}: กฎถูกบันทึกพร้อม Timestamp และ Metadata
    \item \textbf{แสดงผลกราฟ (Visual)}: สร้างกราฟอัตโนมัติสำหรับการตรวจสอบผลลัพธ์
    \item \textbf{Generic และใช้ซ้ำได้}: Engine รองรับข้อมูลตารางทั่วไป ปรับใช้กับ Domain ใหม่ได้ง่าย
\end{enumerate}

\section{ข้อจำกัดและงานในอนาคต}

\subsection{ข้อจำกัดปัจจุบัน}

\begin{enumerate}
    \item \textbf{ขึ้นอยู่กับคุณภาพของ LLM}: การแปลภาษาธรรมชาติเป็น Prolog ขึ้นอยู่กับความสามารถของ Gemini API ซึ่งอาจสร้างกฎที่ไม่ถูกต้องในบางกรณี
    
    \item \textbf{Singleton Variable Warnings}: หากกฎไม่ใช้ตัวแปรบางตัว Prolog จะแสดงคำเตือน ซึ่งต้องแก้ไขด้วยการใช้ underscore (\texttt{\_})
    
    \item \textbf{จำกัดรูปแบบข้อมูล}: ปัจจุบันรองรับเฉพาะข้อมูลตาราง (Tabular Data) ในรูปแบบ CSV เท่านั้น
    
    \item \textbf{Point-based Inference เท่านั้น - ข้อจำกัดสำคัญของ Engine}:
    
    Engine ที่เราพัฒนาขึ้นทำงานแบบ \textit{point-based inference} คือพิจารณาเฉพาะค่าของแต่ละแถวข้อมูล ณ จุดนั้นๆ เท่านั้น \textbf{ไม่สามารถคำนวณย้อนหลังหรือเปรียบเทียบกับจุดข้อมูลก่อนหน้าได้}
    
    \textbf{ตัวอย่างกฎที่ยังไม่รองรับ:}
    \begin{itemize}
        \item ``ถ้าอุณหภูมิลดลงมากกว่า 3 องศาภายใน 3 ชั่วโมง ให้ติดป้ายว่า `ลดลงอย่างรวดเร็ว'"
        \item ``ถ้า PM2.5 เพิ่มขึ้น 20\% จากค่าเฉลี่ยของ 5 ชั่วโมงที่ผ่านมา''
        \item ``ถ้าแนวโน้มเป็นขาขึ้นต่อเนื่อง 4 ช่วงเวลาติดต่อกัน''
    \end{itemize}
    
    กฎเหล่านี้ต้องการ \textit{temporal context} หรือการเข้าถึงข้อมูลก่อนหน้า ซึ่ง Engine ปัจจุบันไม่รองรับเพราะออกแบบมาให้ประมวลผลแต่ละจุดอิสระจากกัน (Stateless row-by-row processing)
    
    \textbf{เหตุผลทางเทคนิค:}
    \begin{itemize}
        \item Prolog facts ถูก assert และ retract ในแต่ละ iteration ไม่มีการเก็บประวัติ
        \item Query string สร้างจากค่าของแถวปัจจุบันเท่านั้น
        \item ไม่มี sliding window หรือ temporal aggregation mechanism
    \end{itemize}
    
    \item \textbf{ไม่มีการตรวจจับความขัดแย้งของกฎ}: หากมีกฎที่ขัดแย้งกัน ระบบจะรายงานทุกป้ายกำกับที่ตรงเงื่อนไข (ใน multi-label mode) หรือใช้กฎแรกที่ตรงเงื่อนไข (ใน single-label mode)
\end{enumerate}

\subsection{การพัฒนาในอนาคต}

\begin{enumerate}
    \item \textbf{รองรับ Temporal Rules และ Historical Context}:
    \begin{itemize}
        \item เพิ่ม Sliding Window Mechanism สำหรับเข้าถึงข้อมูลก่อนหน้า
        \item รองรับการคำนวณ Trend Analysis (ขาขึ้น, ขาลง, อัตราการเปลี่ยนแปลง)
        \item เพิ่ม Temporal Predicates เช่น \texttt{decreased\_by(Variable, Amount, TimeWindow)}
        \item เก็บ Historical Facts ใน Prolog Knowledge Base แทนการ retract ทุกครั้ง
    \end{itemize}
    
    \item \textbf{รองรับรูปแบบข้อมูลเพิ่มเติม}: JSON, ฐานข้อมูล (SQL/NoSQL), Time-series databases
    
    \item \textbf{การปรับแต่งกฎแบบโต้ตอบ (Interactive Rule Refinement)}:
    \begin{itemize}
        \item แสดงตัวอย่างผลลัพธ์แบบ Real-time
        \item เสนอแนะการแก้ไขกฎที่อาจมีปัญหา
        \item Validation feedback loop
    \end{itemize}
    
    \item \textbf{การผสานกับ Active Learning}: ใช้ผลลัพธ์จาก Auto-labeling เป็น Initial labels สำหรับโมเดล ML
    
    \item \textbf{การตรวจจับและแก้ไขความขัดแย้งของกฎ}: 
    \begin{itemize}
        \item วิเคราะห์กฎทั้งหมดเพื่อหาความขัดแย้ง
        \item เสนอแนะลำดับความสำคัญของกฎ (Rule Priority)
    \end{itemize}
    
    \item \textbf{Export กฎไปรูปแบบอื่น}: Decision Trees, Rule-based Neural Networks, PMML
    
    \item \textbf{การเพิ่มประสิทธิภาพ}:
    \begin{itemize}
        \item Parallel processing สำหรับข้อมูลขนาดใหญ่
        \item Caching mechanism สำหรับ Helper Predicates
        \item Incremental labeling (label เฉพาะข้อมูลใหม่)
    \end{itemize}
\end{enumerate}

\section{สรุป}

ระบบนี้แสดงให้เห็นถึงประสิทธิผลของการผสมผสาน Symbolic AI (Prolog) กับ LLMs สมัยใหม่สำหรับการติดป้ายกำกับข้อมูลอัตโนมัติ โดยการให้ผู้เชี่ยวชาญแสดงกฎด้วยภาษาธรรมชาติ ในขณะที่รักษาความเข้มงวดทางตรรกะผ่าน Prolog ระบบนี้เชื่อมช่องว่างระหว่างความสะดวกในการใช้งานกับการให้เหตุผลเชิงรูปแบบ

\subsection{Auto-Labeling Engine ที่พัฒนาขึ้น}

เราได้พัฒนา Auto-Labeling Engine แบบใหม่ที่มีคุณสมบัติสำคัญดังนี้:

\begin{itemize}
    \item \textbf{Generic และ Reusable}: ออกแบบมาสำหรับข้อมูลตารางทั่วไป ไม่ผูกติดกับ domain ใดโดยเฉพาะ
    \item \textbf{Configuration-Driven}: ปรับใช้กับ use case ใหม่ได้โดยแค่แก้ไขไฟล์ JSON
    \item \textbf{Explainable AI}: กฎ Prolog สามารถอ่านและตรวจสอบได้โดยมนุษย์
    \item \textbf{Point-based Inference}: ประมวลผลแต่ละแถวอิสระจากกัน เหมาะสำหรับกฎที่พิจารณาค่า ณ ขณะนั้น
    \item \textbf{Chain Rules Support}: รองรับการให้เหตุผลแบบลำดับชั้นผ่าน Helper Predicates
\end{itemize}

สถาปัตยกรรมแบบเปิด (Open Architecture) ที่ใช้การตั้งค่า JSON ทำให้ระบบสามารถปรับใช้กับหลากหลายสาขาได้ เช่น:
\begin{itemize}
    \item การตรวจสอบคุณภาพอากาศและสิ่งแวดล้อม
    \item การติดตามสุขภาพและอาการทางการแพทย์
    \item การตรวจสอบการปฏิบัติตามกฎระเบียบทางการเงิน
    \item การควบคุมคุณภาพในอุตสาหกรรม
    \item การวิเคราะห์ข้อมูลทางวิทยาศาสตร์
\end{itemize}

\subsection{ข้อจำกัดที่ต้องคำนึงถึง}

ผู้ใช้ควรทราบว่า Engine ปัจจุบัน \textbf{ไม่รองรับกฎที่ต้องการข้อมูลย้อนหลังหรือการเปรียบเทียบกับจุดข้อมูลก่อนหน้า} เช่น กฎที่เกี่ยวกับ:
\begin{itemize}
    \item การเปลี่ยนแปลงในช่วงเวลา (``ลดลง X หน่วยภายใน Y ชั่วโมง'')
    \item การคำนวณค่าเฉลี่ยเคลื่อนที่ (Moving Average)
    \item การวิเคราะห์แนวโน้ม (Trend Analysis)
\end{itemize}

ข้อจำกัดนี้เกิดจากการออกแบบ Point-based Inference ที่ประมวลผลแต่ละแถวอิสระจากกัน การพัฒนาในอนาคตจะเพิ่ม Temporal Context Support เพื่อรองรับกฎประเภทนี้

\subsection{ผลงานหลัก}

งานวิจัยนี้มีส่วนสนับสนุนหลักดังนี้:
\begin{enumerate}
    \item \textbf{Auto-Labeling Engine ใหม่}: ระบบ Generic สำหรับข้อมูลตารางที่ใช้งานง่ายและอธิบายได้
    \item \textbf{การผสานระหว่าง LLM และ Symbolic AI}: เชื่อม Natural Language Interface เข้ากับ Logical Reasoning
    \item \textbf{Configuration-Driven Architecture}: ลดความซับซ้อนในการปรับใช้กับ domain ใหม่
    \item \textbf{Point-based Inference Methodology}: วิธีการที่ชัดเจนและมีประสิทธิภาพสำหรับการติดป้ายกำกับแบบ row-by-row
\end{enumerate}

\section*{กิตติกรรมประกาศ}

งานวิจัยนี้ใช้ Google's Gemini API สำหรับการประมวลผลภาษาธรรมชาติ และ PySwip สำหรับการเชื่อมต่อกับ Prolog

\begin{thebibliography}{9}

\bibitem{pyswip}
PySwip: Python-SWI-Prolog bridge.
\texttt{https://github.com/yuce/pyswip}

\bibitem{gemini}
Google Gemini API Documentation.
\texttt{https://ai.google.dev/}

\bibitem{prolog}
Bratko, I. (2011). \textit{Prolog Programming for Artificial Intelligence}. Addison-Wesley.

\bibitem{dataquality}
Ratner, A., et al. (2017). Snorkel: Rapid training data creation with weak supervision. \textit{VLDB Endowment}, 11(3), 269-282.

\bibitem{xai}
Gunning, D., et al. (2019). XAI—Explainable artificial intelligence. \textit{Science Robotics}, 4(37).

\end{thebibliography}

\end{document}
